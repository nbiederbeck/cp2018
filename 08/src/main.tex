\documentclass{scrartcl}
\usepackage{graphicx}
\usepackage{amsmath}
\usepackage{subcaption}
\usepackage{physics}
\usepackage{subcaption}
\usepackage[
section, % Floats innerhalb der Section halten
below,   % unterhalb der Section aber auf der selben Seite ist ok
]{placeins}

\title{Übungsblatt 08}
\author{%
		Noah Biederbeck, Maximilian Sackel, Jan Spinne
}
\date{Abgabe: 29. Juni 2018}



\begin{document}
\maketitle
\section*{Aufgabe 1: Doppelpendel}
\subsection*{(a)}

Koordinaten:
\begin{align*}
	x_1 &= L_1  \sin{\theta_1} \\
	y_1 &= -L_1  \cos{\theta_1} \\
	x_2 &= L_2  \sin{\theta_2} + x_1 \\
	y_2 &= -L_2  \cos{\theta_2} + y_1 \\
\end{align*}

Ableitungen
\begin{align*}
	\dot{x_1} &= \dot{\theta_1} L_1  \cos{\theta_1} \\
	\dot{y_1} &= \dot{\theta_1 }L_1  \sin{\theta_1} \\
	\dot{x_2} &= \dot{\theta_2 }L_2  \cos{\theta_2} + \dot{x_1} \\
	\dot{y_2} &= \dot{\theta_2} L_2  \sin{\theta_2} + \dot{y_1} \\
\end{align*}


Potentielle Energie:
\begin{align*}
	V &= m_1 g y_1 + m_2 g y_2 \\
	 &= -\left( m_1 + m_2 \right) g  L_1  \cos{\theta_1}-m_2  g  L_2  \cos{\theta_2} \\
\end{align*}


Kinetische Energie:
\begin{align*}
	T &= \frac 1 2 m_1 v_1^2 + \frac 1 2 m_2 v_2^2 \\
	&= \frac 1 2 m_1 \left(\dot{x_1}^2 + \dot{y_1}^2 \right) \frac 1 2 m_2 \left(\dot{x_2}^2 + \dot{y_2}^2 \right) \\
	&= \frac { 1 } { 2 } m _ { 1 } L _ { 1 } ^ { 2 } \dot { \theta } _ { 1 } ^ { 2 } + \frac { 1 } { 2 } m _ { 2 } \left[ L _ { 1 } ^ { 2 } \dot { \theta } _ { 1 } ^ { 2 } + L _ { 2 } ^ { 2 } \dot { \theta } _ { 2 } ^ { 2 } + 2 L _ { 1 } L _ { 2 } \dot { \theta } _ { 1 } \dot { \theta } _ { 2 } \cos \left( \theta _ { 1 } - \theta _ { 2 } \right) \right]  \\
\end{align*}

\subsection*{(b)}
Winkelbeschleunigung in Kleinwinkelnäherung:
\begin{align*}
	\ddot{\theta_1} &= -\frac{g}{L} \left(2 \theta_1 - \theta_2 \right) \\
	\ddot{\theta_2} &= 2 \frac{g}{L} \left( \theta_1 - \theta_2 \right) \\
\end{align*}

Ableiten von $\theta_i = a_i \cos{\omega t}$ und einsetzen liefert:

\begin{align*}
	0 &= - \omega^2 a_1 + \frac g l \left( 2 a_1-a_2\right)  \\
	0 &= - \omega^2 a_2 - 2 \frac g l \left(a_1-a_2\right)  \\
\end{align*}

Umformen nach bspw $a_1$, einsetzen und umformen nach $\omega$ liefert dann:

\begin{equation*}
	 \omega _ { \pm } = \sqrt { \frac { g } { \ell } ( 2 \pm \sqrt { 2 } ) }
\end{equation*}

Test der genäherten Schwingungen sind in Abbildung~\ref{fig:b1} dargestellt.
\begin{figure}[ht]
  \centering
  \begin{subfigure}{0.8\textwidth}
    \centering
    \includegraphics[width=1\linewidth]{build/pendulum_b.png}
  \end{subfigure}

  \begin{subfigure}{0.8\textwidth}
    \centering
    \includegraphics[width=1\linewidth]{build/pendulum_xy_b.png}
  \end{subfigure}
  \caption{Genäherte Schwingungen.}
  \label{fig:b1}
\end{figure}

\subsection*{(c)}
Das Runge Kutta Verfahren 4. Ordnung wird fuer die Startparameter
\begin{align*}
  \theta_1 &= 0.1  & \dot\theta_1 &= 0 \\
  \theta_2 &= \pm \sqrt{2} \theta_1  & \dot\theta_2 &= 0
\end{align*}
durchgefuhrt.

Die Winkel, Winkelgeschwindigkeit, potentielle, kinetische und Gesamtenergie sind in
Abbildungen~\ref{fig:cp} und~\ref{fig:cm} dargestellt.
\begin{figure}[ht]
  \centering
  \begin{subfigure}{0.8\textwidth}
    \centering
    \includegraphics[width=1\linewidth]{build/pendulum_c_plus.png}
    \caption{$\theta_2 = +\sqrt{2}\theta_1$}
    \label{fig:cp}
  \end{subfigure}

  \begin{subfigure}{0.8\textwidth}
    \centering
    \includegraphics[width=1\linewidth]{build/pendulum_c_minus.png}
    \caption{$\theta_2 = -\sqrt{2}\theta_1$}
    \label{fig:cm}
  \end{subfigure}
  \caption{Winkel, Winkelgeschwindigkeit, potentielle, kinetische und Gesamtenergie.}
\end{figure}

\subsection*{(d)}
Die Trajektorien der Pendelmassen sind in Abbildungen~\ref{fig:dp} und~\ref{fig:dm} dargestellt.
\begin{figure}[ht]
  \centering
  \begin{subfigure}{0.8\textwidth}
    \centering
    \includegraphics[width=1\linewidth]{build/pendulum_xy_c_plus.png}
    \caption{$\theta_2 = +\sqrt{2}\theta_1$}
    \label{fig:dp}
  \end{subfigure}

  \begin{subfigure}{0.8\textwidth}
    \centering
    \includegraphics[width=1\linewidth]{build/pendulum_xy_c_minus.png}
    \caption{$\theta_2 = -\sqrt{2}\theta_1$}
    \label{fig:dm}
  \end{subfigure}
  \caption{Trajektorien.}
\end{figure}




\section*{Aufgabe 2: Poincar\'e Schnitt}
\subsection*{(a)}
Die Phasenr\"aume der drei Startbedingungen sind in Abbildung~\ref{fig:2a} dargestellt.
\begin{figure}[ht]
  \centering
  \includegraphics[height=0.96\textheight]{build/poincare_a.png}
  \caption{Phasenr\"aume f\"ur verschiedene Startbedingungen.}%
  \label{fig:2a}
\end{figure}
Es ist zu erkennen, dass sich im periodischen Fall geschlossene Bahnen ergeben.
Hingegen im chaotischen Fall ist der Phasenraum beinahe komplett ausgef\"ullt.

\subsection*{(b)}
Die drei Pendel werden gest\"ort.
Zu jedem Parameter der gegebenes Anfangsbedingungen wird $0.01$ addiert.

Die Abst\"ande der gest\"orten und ungest\"orten L\"osung sind in Abbildung~\ref{fig:2b}
dargestellt.
\begin{figure}[ht]
  \centering
  \includegraphics[height=0.96\textheight]{build/poincare_b.png}
  \caption{Vergleich der gest\"orten und ungest\"orten L\"osung.}%
  \label{fig:2b}
\end{figure}

Es ist eindeutig zu erkennen, dass das periodische und quasi-periodische System stabil ist.
Hingegen ist das chaotische System hochgradig instabil, was an der gro{\ss}en Differenz deutlich
wird.

\subsection*{(c)}
Bei geringen Energien sieht man geschlossene Bahnen im Phasenraum. 
Mit zunehmender Energien werden die Bahnen im Phasenraum chaotischer bis sie
irgendwann nicht mehr zu erkennen sind.
\begin{figure}[ht]
  \centering
  \includegraphics[height=0.96\textheight]{build/poincare_c.png}
  \caption{Poincare-Schnitte bei den Energien 1, 10 und 70 J.}%
  \label{fig:2b}
\end{figure}

\end{document}
