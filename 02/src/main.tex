<<<<<<< HEAD
\documentclass{scrartcl}
\usepackage{graphicx}
\usepackage{amsmath}
\usepackage{subcaption}

\title{Übungsblatt 02}
\author{%
  Noah Biederbeck, Maximilian Sackel, Jan Spinne
}
\date{Abgabe: 04. Mai 2018}



\begin{document}
\maketitle

\section*{Aufgabe 1}

\begin{figure}[h]
  \centering
  \begin{subfigure}{0.48\textwidth}
    \centering
    \includegraphics[width=0.9\linewidth]{build/r1.pdf}
    \caption{}%
    \label{fig:hist_1}
  \end{subfigure}
  \begin{subfigure}{0.48\textwidth}
    \centering
    \includegraphics[width=0.9\linewidth]{build/r2.pdf}
    \caption{}%
    \label{fig:hist_2}
  \end{subfigure}
  \begin{subfigure}{0.48\textwidth}
    \centering
    \includegraphics[width=0.9\linewidth]{build/r3.pdf}
    \caption{}%
    \label{fig:hist_3}
  \end{subfigure}
  \begin{subfigure}{0.48\textwidth}
    \centering
    \includegraphics[width=0.9\linewidth]{build/r4.pdf}
    \caption{}%
    \label{fig:hist_4}
  \end{subfigure}
  \caption{Histogramme der Zufallsgeneratoren.}%
  \label{fig:hist}
\end{figure}

\begin{figure}[h]
  \centering
  \begin{subfigure}{0.48\textwidth}
    \centering
    \includegraphics[width=0.9\linewidth]{build/corr_r1.pdf}
    \caption{}%
    \label{fig:corr_1}
  \end{subfigure}
  \begin{subfigure}{0.48\textwidth}
    \centering
    \includegraphics[width=0.9\linewidth]{build/corr_r2.pdf}
    \caption{}%
    \label{fig:corr_2}
  \end{subfigure}
  \begin{subfigure}{0.48\textwidth}
    \centering
    \includegraphics[width=0.9\linewidth]{build/corr_r3.pdf}
    \caption{}%
    \label{fig:corr_3}
  \end{subfigure}
  \begin{subfigure}{0.48\textwidth}
    \centering
    \includegraphics[width=0.9\linewidth]{build/corr_r4.pdf}
    \caption{}%
    \label{fig:corr_4}
  \end{subfigure}
  \caption{Korrelationen der Zufallsgeneratoren.}%
  \label{fig:corr}
\end{figure}

\section*{Aufgabe 2}
\end{document}
||||||| merged common ancestors
=======
\documentclass{scrartcl}
\usepackage{graphicx}

\title{Übungsblatt 01}
\author{%
  Noah Biederbeck, Maximilian Sackel, Jan Spinne
}
\date{Abgabe: 20. April 2018}


\begin{document}
\maketitle

\section*{Aufgabe 1}
\begin{figure}[h]
  \centering
  \includegraphics[width=0.8\linewidth]{build/plot_01.pdf}
  \caption{Weglänge $N$ aufgetragen gegen mittleren Abstand $r$.}%
  \label{fig:build/plot_01}
\end{figure}
\section*{Aufgabe 2}

\end{document}
>>>>>>> maxi
