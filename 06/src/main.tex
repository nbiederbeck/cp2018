\documentclass{scrartcl}
\usepackage{graphicx}
\usepackage{amsmath}
\usepackage{subcaption}
\usepackage{physics}
\usepackage[
  section, % Floats innerhalb der Section halten
  below,   % unterhalb der Section aber auf der selben Seite ist ok
]{placeins}

\title{Übungsblatt 06}
\author{%
		Noah Biederbeck, Maximilian Sackel, Jan Spinne
}
\date{Abgabe: 15. Juni 2018}



\begin{document}
\maketitle
\section*{Aufgabe 1: Matrixdiagonalisierung}
\subsection*{(a)}
Für das Lanczos-Verfahren wird iterativ folgendes bestimmt:
\begin{align}
  A \vec{q}_{i} &= \beta_{i+1} \vec{q}_{i+1} + \alpha_{i} \vec{q}_{i} + \beta_{i} \vec{q}_{i-1} \\
  \alpha_{i} &= \vec{q}_{i}^{\dagger} A \vec{q}_{i} \\
  \beta_{i+1} &= \vec{q}_{i+1} A \vec{q}_{i}
\end{align}
Es wird die Matrix
\begin{align}
  Q_{N} &= \left\{ \vec{q}_{1} \cdots \vec{q}_{n} \right\}
\end{align}
bestimmt, die die Tridiagonalmatrix liefert:
\begin{align}
  A Q_{n} &= Q_{n} T_{n} \\
  T_{n} &= 
    \left(\begin{matrix}
        \alpha_{1} & \beta_{2}  & 0          & 0         & 0           & \vdots     \\
        \beta_{2}  & \alpha_{2} & \beta_{3}  & 0         & 0           & \vdots     \\
        0          & \beta_{3}  & \alpha_{3} & \beta_{4} & 0           & \vdots     \\
        \cdots     & \cdots     & \cdots     & \cdots    & \beta_{n-1} & \alpha_{_{}n} \\
    \end{matrix}\right) \\
  Q_{n}^{\dagger} A Q_{n} &= T_{n}
\end{align}
Da dies eine Ähnlichkeitstransformation ist, haben $A$ und $T_{n}$ dieselben Eigenwerte.

% So ist mit
% \begin{align}
%   \vec{q}_{0} &= \input{build/q0.tex} \\
%   T_{0} &= \input{build/T0.tex} \\
%   \vec{q}_{1} &= \input{build/q1.tex} \\
%   T_{1} &= \input{build/T1.tex}
% \end{align}

Für mehr siehe \texttt{Mitschrift.pdf} Seite 98.

\subsection*{(b)}
\subsection*{(c)}
\section*{Aufgabe 2: Anharmonischer Oszillator}
\subsection*{(a)}
\subsection*{(b)}
\subsection*{(c)}
\subsection*{(d)}
\end{document}
