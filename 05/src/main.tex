\documentclass{scrartcl}
\usepackage{graphicx}
\usepackage{amsmath}
\usepackage{subcaption}
\usepackage{physics}
\usepackage[
section, % Floats innerhalb der Section halten
below,   % unterhalb der Section aber auf der selben Seite ist ok
]{placeins}

\title{Übungsblatt 05}
\author{%
		Noah Biederbeck, Maximilian Sackel, Jan Spinne
}
\date{Abgabe: 08. Juni 2018}



\begin{document}
\maketitle

\section*{Aufgabe 1: Diagonalisierung per Hand}
Matrix aus der Aufgabe:
\begin{equation}
  \label{eq:M}
  \mathbf{M} = \left(\begin{matrix}
      1 & 1 & 1 & 1 & 1 \\
      1 & 2 & 1 & 1 & 1 \\
      1 & 1 & 3 & 1 & 1 \\
      1 & 1 & 1 & 4 & 1 \\
      1 & 1 & 1 & 1 & 5
  \end{matrix}\right)
\end{equation}
Matrix nach dem Householder-Algorithmus:
\begin{equation}
  \label{eq:H}
  \mathbf{H} = \left(\begin{matrix}
    \input{build/householder_final.txt}
\end{matrix}\right)
\end{equation}
\begin{figure}[ht]
  \centering
  \includegraphics[width=0.8\linewidth]{build/eigenvalues.png}
  \caption{Eigenwerte der Matrix $\mathbf{M}$~\eqref{eq:M}, berechnet durch \texttt{Eigen::EigenSolver()} (blau) und selbstimplementiert (orange).}%
  \label{fig:eigenvalues}
\end{figure}
Es macht keinen Unterschied in der Anzahl Sweeps, ob die Jacobi-Rotation mit $\mathbf{M}$ oder $\mathbf{H}$ durchgefuehrt wird.

\section*{Aufgabe 2: 1D-Hamiltonkette}
\subsection*{(a)}
Zuerst werden Spinzustaende erstellt.
Diese werden in Binaerschreibweise gespeichert ($0 = \downarrow, 1 = \uparrow$) und mit einer eindeutigen Repraesentation verwiesen:
\begin{align*}
  \ket{0 0 1 1} &= \ket{3} \\
  \ket{1 0 0 0} &= \ket{8} \\
\end{align*}

Fuer die Hamiltonmatrix mit $H = \frac{J}{4} \sum_{i}^{N} \left(2 P_{i,i+1} - 1\right)$ wird ueberprueft,
welche Vektoren durch den Permutationsoperator $P_{i,j}$ entstehen:
\begin{align}
  P_{12}\ket{0 1 0 1} &= \ket{1 0 0 1} \\
        P_{12}\ket{5} &= \ket{9} \\
\end{align}
Die entsprechenen Faktoren werden in der Hamiltonmatrix gespeichert.
\begin{figure}[ht]
  \centering
  \includegraphics[width=0.8\linewidth]{build/hamiltonian.png}
  \caption{Hamiltonmatrix der 1D-Hamiltonkette.}%
  \label{fig:hamiltonkette}
\end{figure}

\subsection*{(b)}
Die Eigenwerte der Hamiltonmatrix werden mittels \texttt{Eigen::EigenSolver()} berechnet.
\begin{figure}[ht]
  \centering
  \includegraphics[width=0.8\linewidth]{build/hamiltonian_eigenvalues.png}
  \caption{Eigenwerte der Hamiltonmatrix der 1D-Hamiltonkette.}%
  \label{fig:hamiltonkette_eigenvalues}
\end{figure}

\subsection*{(c)}
Die Laufzeiten der Berechnungen der Eigenwerte werden fuer unterschiedliche Größen bestimmt und gegeneinander aufgetragen.
Der Fit $t\left(N\right) \propto N^x$ passt nicht auf die Werte, da insbesondere der letzte Wert extrem aus der Reihe tanzt.
\begin{figure}[ht]
  \centering
  \includegraphics[width=0.8\linewidth]{build/hamilton_eigenvalues_N.png}
  \caption{%
    Laufzeiten der Diagonalisierung mit \texttt{Eigen::EigenSolver()} für verschiedene Dimensionen der Hamiltonmatrix $N$.
  }%
  \label{fig:build/hamilton_eigenvalues_N}
\end{figure}
\end{document}
