\documentclass{scrartcl}
\usepackage{graphicx}
\usepackage{amsmath}
\usepackage{subcaption}
\usepackage[
section, % Floats innerhalb der Section halten
below,   % unterhalb der Section aber auf der selben Seite ist ok
]{placeins}

\title{Übungsblatt 02}
\author{%
		Noah Biederbeck, Maximilian Sackel, Jan Spinne
}
\date{Abgabe: 04. Mai 2018}



\begin{document}
\maketitle

\section*{Aufgabe 1}
\subsection*{a)}

Der Gleichgewichtszustand tritt bei eine System mit kleineren $q$ schneller ein.
\begin{figure}[ht]
		\centering
		\begin{subfigure}[b]{0.3\textwidth}
				\begin{center}
						\includegraphics[width=\linewidth]{build/q_2_0.pdf}
				\end{center}
				\caption{0 Schritte}
		\end{subfigure}
		\begin{subfigure}[b]{0.3\textwidth}
				\begin{center}
						\includegraphics[width=\linewidth]{build/q_2_30.pdf}
				\end{center}
				\caption{30 Schritte}
		\end{subfigure}
		\begin{subfigure}[b]{0.3\textwidth}
				\begin{center}
						\includegraphics[width=\linewidth]{build/q_2_100.pdf}
				\end{center}
				\caption{10000 Schritte}
		\end{subfigure}
		\caption{Fuer ein zwei Spin Zustand System}%
		\label{fig:1}
\end{figure}

\begin{figure}[ht]
		\centering
		\begin{subfigure}[b]{0.3\textwidth}
				\begin{center}
						\includegraphics[width=\linewidth]{build/q_3_0.pdf}
				\end{center}
				\caption{0 Schritte}
		\end{subfigure}
		\begin{subfigure}[b]{0.3\textwidth}
				\begin{center}
						\includegraphics[width=\linewidth]{build/q_3_250.pdf}
				\end{center}
				\caption{250 Schritte}
		\end{subfigure}
		\begin{subfigure}[b]{0.3\textwidth}
				\begin{center}
						\includegraphics[width=\linewidth]{build/q_3_999.pdf}
				\end{center}
				\caption{10000 Schritte}
		\end{subfigure}
		\caption{Fuer ein drei Spin zustand System}%
		\label{fig:1}
\end{figure}

\subsection*{b)}
In den Plots ist die Magnetisierung gegen die MC-Schritte aufgetragen. 
Ebenso stellt sich die Gleichgewichtsenergie bei kleinen $q$ schneller ein.
\begin{figure}[ht]
		\centering
		\begin{subfigure}[b]{0.49\textwidth}
				\begin{center}
						\includegraphics[width=\linewidth]{build/energy2.pdf}
				\end{center}
				\caption{q=2}
		\end{subfigure}
		\begin{subfigure}[b]{0.49\textwidth}
				\begin{center}
						\includegraphics[width=\linewidth]{build/energy3.pdf}
				\end{center}
				\caption{q=3}
		\end{subfigure}
		\begin{subfigure}[b]{0.49\textwidth}
				\begin{center}
						\includegraphics[width=\linewidth]{build/energy6.pdf}
				\end{center}
				\caption{q=6}
		\end{subfigure}
		\begin{subfigure}[b]{0.49\textwidth}
				\begin{center}
						\includegraphics[width=\linewidth]{build/energy20.pdf}
				\end{center}
				\caption{q=20}
		\end{subfigure}
		\caption{Energie gegen Monte-Carlo Schritte.}%
		\label{fig:1}
\end{figure}

Bei einem $q$-Wert von 20 faellt auf das der Gleichgewichtszustand trotz $10^4$
Monte-Carlo Schritten nicht erreicht wird.
\begin{figure}[ht]
		\centering
		\begin{subfigure}[b]{0.49\textwidth}
				\begin{center}
						\includegraphics[width=\linewidth]{build/magn.pdf}
				\end{center}
				\caption{q=2}
		\end{subfigure}
		\begin{subfigure}[b]{0.49\textwidth}
				\begin{center}
						\includegraphics[width=\linewidth]{build/magn3.pdf}
				\end{center}
				\caption{q=3}
		\end{subfigure}
		\begin{subfigure}[b]{0.49\textwidth}
				\begin{center}
						\includegraphics[width=\linewidth]{build/magn6.pdf}
				\end{center}
				\caption{q=6}
		\end{subfigure}
		\begin{subfigure}[b]{0.49\textwidth}
				\begin{center}
						\includegraphics[width=\linewidth]{build/magn20.pdf}
				\end{center}
				\caption{q=20}
		\end{subfigure}
		\caption{Magnetisierung gegen MC-Schritte.}%
		\label{fig:1}
\end{figure}
\end{document}
