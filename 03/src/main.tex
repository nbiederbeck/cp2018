\documentclass{scrartcl}
\usepackage{graphicx}
\usepackage{amsmath}
\usepackage{subcaption}
\usepackage[
  section, % Floats innerhalb der Section halten
  below,   % unterhalb der Section aber auf der selben Seite ist ok
]{placeins}

\title{Übungsblatt 03.1}
\author{%
  Noah Biederbeck, Maximilian Sackel, Jan Spinne
}
\date{Abgabe: 11. Mai 2018}



\begin{document}
\maketitle

\subsection*{Aufgabe 2}%
\label{sec:aufgabe_2}
Es wird ein $100 \times 100$-Spingitter erstellt.
Dieses wird mit
\begin{enumerate}
  \item zufaelligen Spins
  \item allen Spins = 1
\end{enumerate}
besetzt.
Es werden naechste Nachbarn Wechselwirkungen nach
\begin{equation}
  \mathcal{H} = -J \sum_{i, j \text{n.N.}} s_i s_j
\end{equation}
beruecksichtigt und zufaellige Spins geflippt nach Boltzmann:
\begin{equation}
  p(\texttt{dist\_uniform}) < e^{-\beta \mathcal{H}}.
\end{equation}

Hierbei wird ein Spin geflippt, wenn er sich nach dem Flip in einem energetisch niedrigeren Zustand befindet, oder der Zufall nach Gleichung 2 entscheidet.
\subsection*{a}%
\label{sub:a}
Es werden Momentaufnahmen nach $10^5$ Sweeps fuer $k_B T = 1$ (Abbildung 1) und $K_B T = 3$ (Abbildung 2) erstellt.
\begin{figure}[ht]
  \centering
  % \includegraphics[width=0.8\linewidth]{build/moment_1.png}
  \caption{Momentaufnahme fuer $k_B T = 1$.}
\end{figure}
\begin{figure}[ht]
  \centering
  % \includegraphics[width=0.8\linewidth]{build/moment_3.png}
  \caption{Momentaufnahme fuer $k_B T = 3$.}
\end{figure}

\subsection*{b}
Fuer die Aequilibriumsphase wird
\begin{equation}
  e(t) = \frac{E(t)}{N}
\end{equation}
gegen $t$ fuer verschiedene Temperaturen aufgetragen.
\begin{figure}[ht]
  \centering
  % \includegraphics[width=0.8\linewidth]{build/aequi_1.png}
  \caption{Aequilibrierungsphase fuer $k_b T = 1$.}
\end{figure}
\begin{figure}[ht]
  \centering
  % \includegraphics[width=0.8\linewidth]{build/aequi_2.png}
  \caption{Aequilibrierungsphase fuer $k_b T = 2$.}
\end{figure}
\begin{figure}[ht]
  \centering
  % \includegraphics[width=0.8\linewidth]{build/aequi_3.png}
  \caption{Aequilibrierungsphase fuer $k_b T = 3$.}
\end{figure}
\subsection*{c}
\subsection*{d}
\subsection*{e}







% \section*{Aufgabe 1}
% \subsection*{(a)}
% \begin{align*}
%   s &= \pm 1 \\
%   \mathcal{H} &= -sH \\
%   Z &= \sum_i e^{-\beta \mathcal{H}(i)} \\
%   \beta &= \frac{1}{kT} \\
%   kT &= 1 \Rightarrow \beta = 1 \\
%   Z &= e^{-\mathcal{H}(1)} + e^{\mathcal{H}(2)} \\
%   \left<O\right> &= \sum_{1, 2} \frac{1}{Z} e^{-\beta \mathcal{H} (i)} O(i) \\
%   \intertext{$i$ jeder mgl Zustand}
%                  &= \frac{1}{Z} \left( e^{-H} O(1) + e^H  O(2) \right)\\
%                  &= \frac{1}{e^{-H} + e^H} \left( e^{-H} O(1) + e^H  O(2) \right) \\
%   \intertext{Observable $O$ $\rightarrow$ Magnetisierung $m = s$}
%   \left<m\right> &= \frac{1}{e^{-H} + e^H} \left( e^{-H} \cdot 1 + e^H \cdot (-1) \right)\\
%                  &= \frac{e^{-H} - e^H}{e^{-H} + e^H} \\
%                  &= \frac{-2 \sinh{H}}{2 \cosh{H}} \\
%                  &= -\tanh{H}
% \end{align*}

% \subsection*{(b)}
% Es wird ein zufälliges Magnetfeld gezogen.
% Es werden die Wahrscheinlichkeiten
% \begin{align*}
%   p_{\text{sum}} = e^{\beta \cdot H \cdot (s_{\text{up}} - s)} + e^{\beta \cdot H \cdot (s_{\text{down}} - s)} \\
%   \intertext{mit $s_{\text{up}} = 1$ und $s_{\text{down}} = -1$}
%   p_{\text{up}} = \frac{e^{\beta \cdot H \cdot (s_{\text{up}} - s)}}{p_{\text{sum}}} \\
%   p_{\text{down}} = \frac{e^{\beta \cdot H \cdot (s_{\text{down}} - s)}}{p_{\text{sum}}} \\
% \end{align*}
% berechnet und anschließend der Spin entsprechend ausgerichtet.
% Der Spin wird zufällig geflippt (mit der entsprechenden Wahrscheinlichkeit).
% Dies entspricht einem Schritt.

% Dies wird für 10000 $H$-Werte in jeweils 100000 Schritten simuliert.

% Der Plot 1 zeigt die Wahrscheinlichkeit für die Magnetisierung der entsprechenden Spinzustände.
% Es fällt auf, dass das Ergebnis von dem initialen Spin unabhängig ist.

% Weiterhin richten sich wie erwartet die Spins nach dem Magnetfeld aus.


% \begin{figure}[ht]
%   \centering
%   \includegraphics[width=0.8\linewidth]{build/plot_01_b.png}
%   \caption{%
%     Wahrscheinlichkeiten der Magnetisierung in Abhaengigkeit der Magnetfeldstaerke.%
%   }%
%   \label{fig:plot_01_b}
% \end{figure}



\end{document}
