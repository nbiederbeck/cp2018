\documentclass{scrartcl}
\usepackage{graphicx}
\usepackage{amsmath}

\title{Übungsblatt 01}
\author{%
  Noah Biederbeck, Maximilian Sackel, Jan Spinne
}
\date{Abgabe: 20. April 2018}


\begin{document}
\maketitle

\section*{Aufgabe 1}
Random Walk mit $N = 10, 15, \ldots, 60$ Schritten,
mittlerer Abstand $r$ aus $C = 10^5$ Walks.
\begin{figure}[h]
  \centering
  \includegraphics[width=0.8\linewidth]{build/plot_01.pdf}
  \caption{Weglänge $N$ aufgetragen gegen mittleren Abstand $r$.}%
  \label{fig:build/plot_01}
\end{figure}

\section*{Aufgabe 2}
\subsection*{a.}
Bestimme zuerst $\pi$:
Hierfür werden gleichverteilte Zufallszahlen in 3 Dimensionen
zwischen 0..1 gezogen.
Ist der Abstand zum Ursprung größer, als der Kugelradius,
werden sie verworfen:
\begin{equation}
  r^2 = x^2 + y^2 + z^2 \le 1
\end{equation}

Bestimme das Kugelvolumen aus dem Verhältnis aller gezogenen Zahlen $N$
und der Zahlen innerhalb der Einheitskugel $n$:
\begin{align}
  8 \cdot \frac{n}{N} &= \frac{4}{3} \pi r^3 \\
  \Rightarrow 6 \frac{n}{N} &= \pi
\end{align}

\subsection*{b.}
Laut Wolfram Alpha ist
\begin{align}
  \int_{-\infty}^{T_i} \frac{1}{\sqrt{\pi}} \exp{\left(-x^2\right)} dx&\\
  \intertext{für $T_i = \left\{-1, 1.1631, \infty\right\}$}
  &= \left\{0.078, 0.95, 1\right\}.\\
  \intertext{Eine MC-Integration mit $N = 10^9$ Punkten liefert}
  &= \left\{0.078, 0.05, 0\right\}.
\end{align}


\end{document}
