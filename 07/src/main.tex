\documentclass{scrartcl}
\usepackage{graphicx}
\usepackage{amsmath}
\usepackage{subcaption}
\usepackage{physics}
\usepackage[
section, % Floats innerhalb der Section halten
below,   % unterhalb der Section aber auf der selben Seite ist ok
]{placeins}

\title{Übungsblatt 07}
\author{%
		Noah Biederbeck, Maximilian Sackel, Jan Spinne
}
\date{Abgabe: 22. Juni 2018}



\begin{document}
\maketitle
\section*{Aufgabe 1: Harmonischer Oszillator}
\subsection*{a)}
Beim Euler Algor werden aus der Differentialgleichung zweiter Ordnung
\begin{equation}
		m \ddot{r} = F(r)
\end{equation}
zwei lineare DGLs
\begin{eqnarray}
		\dot{r} &= v,  \\
		\dot{v} &= \frac{1}{m} F(r). \\ 
\end{eqnarray}
Diese werden ueber den Ansatz 
\begin{equation}
		r(i+1) = r(i) + h f(t_i, x_i)
\end{equation}
geloest. Dabei wird die Zeit in diskretisiert und deren Schrittweite 
betraegt $h$.
\begin{figure}[ht]
		\centering
		\includegraphics[width=0.8\linewidth]{build/euler.pdf}
		\caption{Euler Algo}
		\label{fig:euler}
\end{figure}
\end{document}
