\documentclass{scrartcl}
\usepackage{graphicx}
\usepackage{amsmath}
\usepackage{subcaption}
\usepackage{physics}
\usepackage[
section, % Floats innerhalb der Section halten
below,   % unterhalb der Section aber auf der selben Seite ist ok
]{placeins}

\title{Übungsblatt 07}
\author{%
  Noah Biederbeck, Maximilian Sackel, Jan Spinne
}
\date{Abgabe: 22. Juni 2018}



\begin{document}
\maketitle
\section*{Aufgabe 1: Harmonischer Oszillator}
\subsection*{(a)}
Beim Euler Algorithmus werden aus der Differentialgleichung zweiter Ordnung
\begin{equation}
  m \ddot{r} = F(r)
\end{equation}
zwei lineare DGLs
\begin{eqnarray}
  \dot{r} &= v,  \\
  \dot{v} &= \frac{1}{m} F(r). \\
\end{eqnarray}
Diese werden über den Ansatz
\begin{equation}
  r(i+1) = r(i) + h f(t_i, x_i)
\end{equation}
gelöst. Dabei wird die Zeit in diskretisiert und die Schrittweite
beträgt $h$.
In Abbildungen~\ref{fig:euler} bis~\ref{fig:rk4}
sind die Amplituden der Geschwindigkeit und Auslenkung (in $x$-Richtung) im oberen, die Energie im unteren Plot gegen die Zeitschritte aufgetragen.
Es ist zu erkennen, dass bei allen Verfahren die maximale Auslenkung jedes Mal erreicht wird, und
die Energie erhalten ist.
Es wird für eine Zeit von $T = 4\pi$ eine Schrittweite von $h = \frac{\pi}{10000}$ gewählt,
sodass $N = 40000$ Simulationsschritte durchgeführt werden.
\begin{figure}[ht]
  \centering
  \includegraphics[width=0.8\linewidth]{build/euler.png}
  \caption{Euler}%
  \label{fig:euler}
\end{figure}
\begin{figure}[ht]
  \centering
  \includegraphics[width=0.8\linewidth]{build/runge2.png}
  \caption{Runge Kutta 2. Ordnung}%
  \label{fig:rk2}
\end{figure}
\begin{figure}[ht]
  \centering
  \includegraphics[width=0.8\linewidth]{build/runge4.png}
  \caption{Runge Kutta 4. Ordnung}%
  \label{fig:rk4}
\end{figure}
\subsection*{(b)}
In Abbildungen~\ref{fig:b1} bis~\ref{fig:b3} sind verschiedene Anfangsbedingungen getestet.
\begin{figure}[ht]
  \centering
  \includegraphics[width=0.8\linewidth]{build/b1.png}
  \caption{$r(0)$ beliebig, $v(0) = 0$}%
  \label{fig:b1}
\end{figure}
\begin{figure}[ht]
  \centering
  \includegraphics[width=0.8\linewidth]{build/b2.png}
  \caption{$r(0)$ beliebig, $v(0)$ beliebig}%
  \label{fig:b2}
\end{figure}
\begin{figure}[ht]
  \centering
  \includegraphics[width=0.8\linewidth]{build/b3.png}
  \caption{$r(0)$ senkrecht $v(0)$}%
  \label{fig:b3}
\end{figure}

\section*{Aufgabe 2: Kepler-Ellipsen}
Für bessere Darstellungsmöglichkeiten wird im Folgenden $z = 0$ gehalten.
\subsection*{(a)}

In Abbildung~\ref{fig:kepler} (links) ist die Bahn eines Planeten um die Sonne dargestellt.
Es werden $T= 24$h simuliert mit einer Schrittweite von $h = \frac{1}{5000}$.
Es ist $r(0) = (1, 0, 0)$ und $v(0) = (0.27, 1, 0)$
\begin{figure}[ht]
  \centering
  \includegraphics[width=0.8\linewidth]{build/kepler_rk.png}
  \caption{Planetenbahn und Energie-/Impulserhaltung.}%
  \label{fig:kepler}
\end{figure}
\subsection*{(b)}
In Abbildung~\ref{fig:kepler} (rechts) ist ausserdem die Energie und der Drehimpuls aufgetragen.
Offenbar ist beides erhalten.
\subsection*{(c)}
Das Potential wird geändert zu
\begin{equation}
  V(r) = \frac{-mG}{r^\alpha}
\end{equation}
mit $\alpha = 0.9$ (Abbildung~\ref{fig:2c09}),
und $\alpha = 1.1$ (Abbildung~\ref{fig:2c11}).
\begin{figure}[ht]
  \centering
  \includegraphics[width=0.8\linewidth]{build/2c09.png}
  \caption{Planetenbahn und Energie-/Impulserhaltung für $\alpha = 0.9$}%
  \label{fig:2c09}
\end{figure}
\begin{figure}[ht]
  \centering
  \includegraphics[width=0.8\linewidth]{build/2c11.png}
  \caption{Planetenbahn und Energie-/Impulserhaltung für $\alpha = 1.1$}%
  \label{fig:2c11}
\end{figure}
\subsection*{(d)}
In einem 3-Körper-Problem wirkt die Gravitation zwischen Erde-Sonne, Erde-Mond, Mond-Sonne.
Deswegen wird die wirkende Kraft
\begin{align}
  r &= r_2 - r_1 \\
  F &= - G \cdot \left(\frac{- m_1 \cdot m_2 \cdot r}{r^3} + \frac{m_1 \cdot m_{\text{sun}} \cdot
    r_1}{r_1^3}\right)\text{,}
\end{align}
siehe \texttt{src/rungekutta\_moon.h}.
Hierbei ist der Index ${}_1$ der "eigene" Körper,
der Index ${}_2$ der "andere".
\subsection*{(e)}
Die Bahnkurven des Planeten und des Mondes sind mit laut Aufgabenzettel gegebenen Startparametern
in Abbildung~\ref{fig:kepler_moon} dargestellt.
\begin{figure}[ht]
  \centering
  \includegraphics[width=0.8\linewidth]{build/kepler_moon.png}
  \caption{Bahnkurven von Planet und Mond.}%
  \label{fig:kepler_moon}
\end{figure}

Die Bahn des Mondes relativ zum Planeten ist in Abbildung~\ref{fig:moon_planet} dargestellt.
\begin{figure}[ht]
  \centering
  \includegraphics[width=0.8\linewidth]{build/moon_planet.png}
  \caption{Bahn des Mondes Relativ zum Planeten.}%
  \label{fig:moon_planet}
\end{figure}

Offensichtlich erreicht unser Programm keine stabile Bahn mit gegebenen Parametern, der Mond
stürzt nach 24 Stunden in die Sonne.

Die Bahnkurve des Planeten mit Mond ist anders als die ohne Mond, da dieser gravitative Einflüsse auf den
Planeten ausübt.
\subsection*{(f)}
\end{document}
