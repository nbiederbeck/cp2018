\documentclass{scrartcl}
\usepackage{graphicx}
\usepackage{amsmath}
\usepackage{subcaption}
\usepackage{physics}
\usepackage{subcaption}
\usepackage[
section, % Floats innerhalb der Section halten
below,   % unterhalb der Section aber auf der selben Seite ist ok
]{placeins}

\title{Übungsblatt 08}
\author{%
		Noah Biederbeck, Maximilian Sackel, Jan Spinne
}
\date{Abgabe: 13. Juli 2018}



\begin{document}
\maketitle
\section*{Aufgabe 1: 2D Lennard-Jones Fluid}
Etwas wie das hier tun....
\begin{align*}
  \mathbf{\underline{A}} \cdot
  \left(\begin{matrix}
      r_x \\
      r_y \\
      v_x \\
      v_y \\
      r_x^{'} \\
      r_y^{'} \\
      v_x^{'} \\
      v_y^{'} \\
  \end{matrix}\right)
  &=
  \left(\begin{matrix}
      2 r_x - r_x^{'} + a_x h^2 \\
      2 r_y - r_y^{'} + a_y h^2 \\
      \\
      \\
      r_x \\
      r_y \\
      v_x \\
      v_y \\
  \end{matrix}\right)
\end{align*}
Mit $r$ aktuell und $r^{'}$ der letzte...


\subsection*{(a)}

Ist
\begin{align*}
  V(r) &= 4 \cdot \left[\left(\frac{1}{r}\right)^{12} - \left(\frac{1}{r}\right)^{6}\right] \\
       &= 4 \cdot \left[{r}^{-12} - {r}^{-6}\right] \\
  \intertext{so ist}
  F(r) &= - \nabla V \\
       &= - 4 \cdot \left[-12 {r}^{-13} + 6 {r}^{-7}\right] \\
       &= - 4 \cdot \left[- 12 \left(\frac{1}{r}\right)^{13} + 6 \left(\frac{1}{r}\right)^{7}\right] \\
\end{align*}

Außerdem ist
\begin{align*}
  \dot{r} &= v \\
  \ddot{r} &= \dot{v} = a = \frac{1}{m} \cdot F(r)
\end{align*}

Algorithmus ist also
\begin{align*}
  \left(\begin{matrix} r_x \\ r_y \\ v_x \\ v_y \\ \end{matrix}\right)_{n+1} = todo
\end{align*}

\end{document}
